Для определения в точке $r'$ решения $P(r',t')$ уравнения \ref{eq:wav} введём локальное время $t^* = t - t' + \frac{|r'-r|}{c}$. \cite{samarskii}
Перейдём к сферической системе координат
$$
	u(r,\varphi,\theta,t) = P(r\cos \varphi \cos \theta ,r\sin \varphi \cos \theta,r \sin \theta,t).
$$
$$
	U(r,\varphi,\theta,t) = u(r,\varphi,\theta,t^*).
$$
 Тогда оператор Лапласа будет иметь вид:
\begin{equation}
	\Delta u={\frac{\partial^{2}u}{\partial r^{2}}}+{\frac{2}{r}}{\frac{\partial u}{\partial r}}+{\frac{1}{r^{2}\sin\theta}}{\frac{\partial}{\partial\theta}}\left(\sin\theta{\frac{\partial u}{\partial\theta}}\right)+{\frac{1}{r^{2}\sin^{2}\theta}}{\frac{\partial^{2}u}{\partial\varphi^{2}}}
\end{equation}
Выразим производные функции $u$ через производные $U$:
\begin{gather}
	u_{r}=U_{r^{*}}+{\frac{1}{a}}\,U_{t^{*}}, \\
	u_{r r}=U_{r^{*}r^*}+{\frac{2}{a}}\,U_{r^{*}t}+{\frac{1}{a^{2}}}\,U_{t^{*}t},\\
	u_\theta = U_\theta \\
	u_\varphi = U_\varphi \\
	u_t = U_{t^*} \\
	u_{\theta\theta} = U_{\theta\theta} \\
	u_{\varphi\varphi} = U_{\varphi\varphi} \\
	u_{tt} = U_{t^*t^*} 
\end{gather}
Уравнение \ref{eq:wav} переходит в 
\begin{equation}
	\Delta U=-\frac{2}{a r^{*}}\frac{\partial}{\partial r^{*}}\,(r^{*}U_{t^{*}})-F\,(r^{*},\,\theta^{*},\,t^{*}).
\end{equation}
Рассматривая уравнение как уравнение Лапласа с параметром $t^*$ воспользуемся формулой Грина \cite{samarskii}:
\begin{equation}
	4\pi U\left(M_{0},\,0\right)=\iint_S\left[\frac{1}{r^{\ast}}\,\frac{\partial U}{\partial n}-U\,\frac{\partial}{\partial n}\left(\frac{1}{r^{\ast}}\right)\right]dS + \iiint_T \frac{2}{c r^{*2}} \diff{}{r^*} \left(r^*\diff{U}{t^*}\right) d\tau +
	\iiint_T \frac{F}{r^*} d\tau
\end{equation}
Обозначим $$I = \iiint_T \frac{2}{cr^{*2}} \diff{}{r^*} \left(r^*\diff{U}{t^*}\right) d\tau.$$
Так как $r^*=0$ является сингулярной точкой, вычислим предел
\begin{equation}
	I = \lim\limits_{\varepsilon\to 0} \iiint_{T \setminus T_\varepsilon} \frac{2}{cr^{*2}} \diff{}{r^*} \left(r^*\diff{U}{t^*}\right) d\tau,
\end{equation}
где $T_\varepsilon$ -- шар радиуса $\varepsilon$. Подставим якобиан сферической сисетмы координат
\begin{equation}
	d\tau = r^{*2} \sin \theta\, dr\,d\theta \, d\varphi,
\end{equation}
\begin{equation}
	I = \lim\limits_{\varepsilon\to 0} \iiint_{T \setminus T_\varepsilon} \frac{2}{c} \diff{}{r^*} \left(r^*\diff{U}{t^*}\right) \sin \theta\, dr\,d\theta \, d\varphi.
\end{equation}
Проинтегрируем по $r$
\begin{equation}
	I =  \iint_{S} \frac{2}{cr^*} \left(\diff{U}{t^*}\right) \cos(n,r) dS - \lim\limits_{\varepsilon\to 0} \iint_{S_\varepsilon} \frac{2}{cr^*} \left(\diff{U}{t^*}\right) \cos(n,r) dS
\end{equation}
Так как радиус $T_\varepsilon$ стремится к нулю, то и предел в предыдущем выражении также равен нулю.
Таким образом получаем интегральную формулу 
\begin{equation}
	u\left(M_{0},t_{0}\right)=\frac{1}{4\pi}\iint_{S}^{}\left\{\frac{1}{r}\left[\frac{\partial u}{\partial n}\right]-\left[u\right]\frac{\partial}{\partial n}\left(\frac{1}{r}\right)+\frac{1}{a r}\left[\frac{\partial u}{\partial t}\right]\frac{d r}{d n}\right\}d S_{M}+\iiint_T\frac{[f]}{r}\,d\tau
	\label{eq:kirsam}
\end{equation}

