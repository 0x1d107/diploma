	\begin{comment}
	\subsection{Приближение Борна}
	Представим скорость распространения волны в виде
	\begin{equation}
		\frac{1}{c^2(r)} = \frac{1+a(r)}{c_1^2}.
	\end{equation}
	Обозначим значение, обратное по величине скорости волны, как медленность 
	\begin{gather*}
		s = \frac{1}{c} \\
		s_1 = \frac{1}{c_1} \\
		s_2 = \frac{1}{c_2}
	\end{gather*}
	Таким образом функция $a(r)$ является квадратом нормированной аномальной медленности:
	$$
	a(r) = \frac{s^2 - s_1^2}{s_1^2} = \frac{\Delta s^2}{s_1^2}
	$$
	Падающее волновое поле удовлетворяет волновому уравнению в области $D_1$:
	\begin{equation}
		\nabla^2 P_0 - \frac{1}{c_1^2}\diff[2]{P_0}{t} = 0.
		\label{eq:wavp0}
	\end{equation}
	Преломлённое волновое поле также является решением волнового уравнения, но в области $D_2$ с другой скоростью звука:
	\begin{equation}
		\nabla^2 P_2 - \frac{1}{c_2^2}\diff[2]{P_2}{t} = 0.
		\label{eq:wavp2}
	\end{equation}
	Из условия \ref{eq:rc2} непрерывности значения  волнового поля на границе однородной области можно переписать уравнение \ref{eq:wavp2} в виде
	\begin{equation}
		\nabla^2 P_0 + \nabla^2 P_1 - \frac{1}{c_2^2}\diff[2]{P_0}{t} - \frac{1}{c_2^2}\diff[2]{P_1}{t}  = 0.
		\label{eq:wavp01sum}
	\end{equation}
	Подставим в выражение $\nabla^2 P_0$ из уравнения \ref{eq:wavp0}, и
	тогда отражённое волновое поле будет удовлетворять уравнению \cite{bleistein2012mathematical}
	\begin{equation}
		\nabla^2 P_1 - \frac{1}{c_2^2}\diff[2]{P_1}{t}  = - \frac{1}{c_1^2}\diff[2]{P_0}{t} + \frac{1}{c_2^2}\diff[2]{P_0}{t}.
	\end{equation}
	Так как отражённая волна распространяется в области $D_1$, то она удовлетворяет следующему неоднородному волновому уравнению:
	\begin{equation}
		\nabla^2 P_1 - \frac{1}{c_1^2}\diff[2]{P_1}{t}  = \left(\frac{1}{c_2^2} - \frac{1}{c_1^2}\right)\left(\diff[2]{P_0}{t} + \diff[2]{P_1}{t}\right).
		\label{eq:wavp1inh}
	\end{equation}
	Используя введённые ранее обозначения, перепишем уравнение \ref{eq:wavp1inh} следующим образом:
	\begin{equation}
		\nabla^2 P_1 - \frac{1}{c_1^2}\diff[2]{P_1}{t}  = \Delta s^2 \cdot \diff[2]{P_2}{t}.
	\end{equation}
	Следовательно, решение этого неоднородного дифференциального уравнения можно выразить через функцию Грина:
	\begin{equation}
		P_1(r',t')   = \iiint\limits_{D_1} \int\limits_{-\infty}^{\infty} \Delta s^2 \cdot \left(\diff[2]{P_0(r,t)}{t}+\diff[2]{P_1(r,t)}{t}\right) G(r',t'|r,t)\, dt \, dv.
	\end{equation}
	Приближение Борновского типа состоит в том, чтобы положить отражённое поле под знаком интеграла равным нулю:
	\begin{equation}
		P_1(r',t')   = \iiint\limits_{D_1} \int\limits_{-\infty}^{\infty} \Delta s^2 \cdot 	\diff[2]{P_0(r,t)}{t} G(r',t'|r,t)\, dt \, dv.
	\end{equation}
\end{comment}
%DONE: !!! Оттражение от наклонной плоскости!!
\begin{comment}
	\subsection{Отражение от наклонной плоской границы}
	Если подставить в выражение \ref{eq:kir} функцию Грина \ref{eq:green3d} в явном виде и проинтегрировать по времени, то получится \cite{magisters}
	\begin{align}
		P(r',t') =& \frac{1}{4\pi} \iint_B	\left[ \frac{1}{c|r'-r|}\partial_n|r'-r|\frac{\partial \tilde{P}(r,t')}{\partial t'} \right.\nonumber \\
		&\left. - \partial_n
		\left(\frac{1}{|r'-r|}\right)\tilde{P}(r,t') +\frac{1}{|r'-r|}\partial_n \tilde{P}(r,t')\right]\,ds
		\label{eq:kirgreen3d}
	\end{align}
	где $\tilde{P}(r,t) = P\left(r,t - \frac{|r'-r|}{c}\right)$ -- волна запаздывания.
	
	Пусть поверхность $B$ является плоскостью, заданной уравнением 
	$$
	Ax+By+Cz=D,
	$$ внешняя нормаль которой направлена вдоль вектора $n =(A,B,C)$, то есть $\partial_n = n \cdot \nabla$. Тогда можно аналитически вычислить значение нормальных производных
	\begin{eqnarray}
		\partial_n|r'-r| =&\frac{A(x' - x)}{|n|\cdot|r' - r|}+\frac{B(y' - y)}{|n|\cdot |r' - r|} + \frac{C(z' - z)}{|n|\cdot|r' - r|}, \\ 
		\partial_n\left(\frac{1}{|r'-r|}\right) =&  - \frac{A(x' - x)}{|n|\cdot|r' - r|^3}-\frac{B(y' - y)}{|n|\cdot |r' - r|^3} - \frac{C(z' - z)}{|n|\cdot|r' - r|^3}.
	\end{eqnarray} 
	Значения нормальной и временной производных от волны запаздывания можно вычислить численно, заменив производную на её разностный аналог, либо аналитически, если функция волны запаздывания также задана аналитически.
	
	Например, пусть волна на отражающей поверхности задана функцией
	\begin{equation}
		P(r,t) = \left( 1-2\,{\pi}^{2}{t}^{2} \right) {{\rm e}^{-{\pi}^{2}{t}^{2}}},
	\end{equation}
	тогда волна запаздывания будет иметь вид
	\begin{equation}
		\tilde{P}(r,t) = \left( 1-2\,{\pi}^{2} \left( t-{\frac {\sqrt {{x}^{2}+{y}^{2}+{z}^{2}
			}}{c}} \right) ^{2} \right) {{\rm e}^{-{\pi}^{2} \left( t-{\frac {
						\sqrt {{x}^{2}+{y}^{2}+{z}^{2}}}{c}} \right) ^{2}}}.
		\label{eq:latewavrick}
	\end{equation}
	Для удобства, обозначим $|r| = \sqrt{x^2+y^2+z^2}$. Производная по времени выражается как
	\begin{align}
		\diff{\tilde{P}(r,t)}{t}=&-4\,{\pi}^{2} \left( t-{\frac { \left| r \right| }{c}} \right) {
			{\rm e}^{-{\pi}^{2} \left( t-{\frac { \left| r \right| }{c}} \right) ^
				{2}}}\nonumber\\
		&-2\, \left( 1-2\,{\pi}^{2} \left( t-{\frac { \left| r \right| }{c
		}} \right) ^{2} \right) {\pi}^{2} \left( t-{\frac { \left| r \right| 
			}{c}} \right) {{\rm e}^{-{\pi}^{2} \left( t-{\frac { \left| r \right| 
					}{c}} \right) ^{2}}},
	\end{align}
	а производную по пространственной переменной $x$ можно вычислить следующим образом:
	\begin{align}
		\diff{\tilde{P}(r,t)}{x} = -4\,{\frac { \left( -2\, \left| r \right| {\pi}^{2}ct+ \left( {c}^{2}{
					t}^{2}+|r|^2 \right) {\pi}^{2}-3/2\,{c}^{2} \right) 
				\left( tc- \left| r \right|  \right) x{\pi}^{2}}{|r|{c}^{4}}{{\rm e}^{-{\frac {{\pi}^{2} \left( -tc+ \left| r
							\right|  \right) ^{2}}{{c}^{2}}}}}}.
		\label{eq:latewavrickdx}
	\end{align}
	
	Так как функция \ref{eq:latewavrick} симметрична относительно пространственных переменных, то частные производные $\diff{\tilde{P}(r,t)}{y}$ и $\diff{\tilde{P}(r,t)}{z}$
	будут иметь вид, аналогичный выражению \ref{eq:latewavrickdx}.  
	Значение производной вдоль внешней нормали плоскости $B$ можно вычислить как 
	\begin{equation}
		\diff{\tilde{P}(r,t)}{n} = \frac{A}{|n|} \diff{\tilde{P}(r,t)}{x} +
		\frac{B}{|n|} \diff{\tilde{P}(r,t)}{y} +
		\frac{C}{|n|} \diff{\tilde{P}(r,t)}{z},
	\end{equation}
	где $|n| = \sqrt{A^2+B^2+C^2}$.
	%Подставим значения производных в выражение \ref{eq:kirgreen3d}
	%\begin{equation}
	%	P(r',t') = \frac{1}{4\pi} \iint\limits_B
	%	\left[\frac{z'-z}{|r'-r|^2}\left(\frac{1}{c}\frac{\partial\tilde{P}(r',t')}{\partial t'}+\frac{\tilde{P}(r',t')}{|r'-r|}\right) - \frac{1}{|r'-r|}\frac{\partial\tilde{P}(r',t')}{\partial z}\right]\,ds
	%\end{equation}
	
\end{comment}